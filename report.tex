\documentclass[conference]{IEEEtran}
\IEEEoverridecommandlockouts

\usepackage{cite}
\usepackage{amsmath,amssymb,amsfonts}
\usepackage{algorithmic}
\usepackage{graphicx}
\usepackage{textcomp}
\usepackage{xcolor}
\usepackage{listings}
\usepackage{hyperref}
\usepackage{booktabs}
\usepackage{multirow}
\usepackage{array}

\lstset{
    basicstyle=\ttfamily\footnotesize,
    breaklines=true,
    frame=single,
    language=Java,
    keywordstyle=\color{blue},
    commentstyle=\color{green!60!black},
    stringstyle=\color{orange},
    numbers=left,
    numberstyle=\tiny\color{gray},
    numbersep=5pt
}

\def\BibTeX{{\rm B\kern-.05em{\sc i\kern-.025em b}\kern-.08em
    T\kern-.1667em\lower.7ex\hbox{E}\kern-.125emX}}

\begin{document}

\title{SunDevilSync 2.0: A Blockchain-Based Event Management and Credential Verification Platform for Higher Education}

\author{
\IEEEauthorblockN{Team Members}
\IEEEauthorblockA{\textit{Arizona State University} \\
Tempe, Arizona, USA}
}

\maketitle

\begin{abstract}
This paper presents SunDevilSync 2.0, a decentralized event management platform that leverages blockchain technology to issue verifiable, tamper-proof credentials for student event participation. Built on the Polygon network, the system combines traditional web-based event management with smart contract-based NFT badges (ERC-721) and a gamified reward token system (ERC-20). The platform addresses key challenges in credential verification, including fraud prevention, portability, and permanent accessibility. We implement soulbound tokens (SBTs) for non-transferable achievement credentials and tradable collectible NFTs for special accomplishments. Our hybrid architecture stores sensitive user data off-chain while maintaining immutable credential records on-chain, balancing privacy requirements with transparency. The system demonstrates practical blockchain integration in educational contexts, providing students with cryptographically verifiable proof of participation that employers and institutions can validate without intermediaries.
\end{abstract}

\begin{IEEEkeywords}
Blockchain, Smart Contracts, NFT, Soulbound Tokens, ERC-721, ERC-20, Credential Verification, Higher Education, Polygon, Event Management
\end{IEEEkeywords}

\section{Introduction}

The verification of educational credentials and extracurricular achievements presents significant challenges in traditional systems. Paper certificates are easily forged, centralized databases can be compromised, and verification processes often require contacting issuing institutions directly---a time-consuming and unreliable process. According to the Association of Certified Fraud Examiners, credential fraud costs organizations billions annually, with educational credential fraud being a growing concern \cite{acfe2022}.

Blockchain technology offers a compelling solution through its inherent properties of immutability, transparency, and decentralization. Non-fungible tokens (NFTs) provide a mechanism for issuing unique, verifiable digital credentials that cannot be duplicated or falsely claimed. When combined with the concept of soulbound tokens---non-transferable NFTs permanently bound to a recipient's wallet---we achieve a system where achievements are both verifiable and non-transferable.

SunDevilSync 2.0 is a comprehensive platform designed for Arizona State University that enables:

\begin{itemize}
    \item Students to discover, register for, and attend campus events
    \item Automatic issuance of blockchain-verified attendance badges
    \item A gamified token reward system (SDC tokens) for participation
    \item Conversion of earned tokens to cryptocurrency (MATIC)
    \item Public verification of credentials by employers and faculty
\end{itemize}

This paper details the technical architecture, smart contract implementation, security considerations, and analysis of our blockchain-based credential verification system.

\section{Prior Work and Literature Review}

\subsection{Blockchain in Education}

The application of blockchain technology in education has been extensively explored. MIT's Digital Credentials Consortium developed a framework for issuing and verifying digital credentials using blockchain \cite{mit2019}. Similarly, the European Union's EBSI (European Blockchain Services Infrastructure) includes educational credential verification as a key use case \cite{ebsi2021}.

\subsection{Non-Fungible Tokens for Credentials}

NFTs have emerged as a powerful mechanism for representing unique digital assets. OpenBadges, developed by Mozilla Foundation, established standards for digital credentials that influenced blockchain-based approaches \cite{openbadges}. However, traditional NFTs are transferable, which poses challenges for credentials that should remain bound to their original recipients.

\subsection{Soulbound Tokens}

Vitalik Buterin's proposal for ``Soulbound'' tokens addresses the transferability concern by introducing non-transferable NFTs that represent commitments, credentials, and affiliations \cite{buterin2022}. Our implementation of AchievementSBT directly applies this concept to educational credentials.

\subsection{Layer 2 Solutions}

Polygon (formerly MATIC) provides a Layer 2 scaling solution for Ethereum that enables fast, low-cost transactions while maintaining security guarantees through periodic checkpointing to the Ethereum mainnet \cite{polygon2021}. This makes it ideal for applications requiring frequent, low-value transactions like credential issuance.

\subsection{Gamification in Education}

Research by Hamari et al. demonstrates that gamification elements, including points and badges, significantly increase engagement in educational contexts \cite{hamari2014}. Our SDC token system applies these principles to incentivize event participation.

\section{Technical Architecture}

\subsection{System Overview}

SunDevilSync 2.0 employs a hybrid architecture combining traditional web technologies with blockchain infrastructure. Figure~\ref{fig:architecture} illustrates the high-level system design.

\begin{figure}[htbp]
\centerline{\fbox{\parbox{0.9\columnwidth}{
\textbf{System Architecture}\\[0.5em]
\small
\texttt{+---------------------------+}\\
\texttt{|    Frontend (HTML/JS)     |}\\
\texttt{+------------+-------------+}\\
\texttt{             |              }\\
\texttt{+------------v--------------+}\\
\texttt{|   Backend (Node/Express)  |}\\
\texttt{+------+----------+---------+}\\
\texttt{       |          |          }\\
\texttt{+------v----+ +---v---------+}\\
\texttt{| SQLite DB | | Polygon     |}\\
\texttt{|           | | Blockchain  |}\\
\texttt{+-----------+ +-------------+}
}}}
\caption{High-level system architecture showing the three-tier design with blockchain integration.}
\label{fig:architecture}
\end{figure}

\subsection{Component Architecture}

The system comprises four main layers:

\textbf{Presentation Layer:} HTML/CSS/JavaScript frontend providing user interfaces for event browsing, badge galleries, and administrative functions.

\textbf{Application Layer:} Node.js/Express backend handling authentication, business logic, and blockchain interactions through ethers.js.

\textbf{Data Layer:} SQLite database storing user profiles, event metadata, and transaction references.

\textbf{Blockchain Layer:} Smart contracts deployed on Polygon Amoy testnet handling token minting, badge issuance, and credential verification.

\subsection{Network Configuration}

We deploy on Polygon Amoy testnet (Chain ID: 80002) for development, with production deployment planned for Polygon mainnet (Chain ID: 137). Key network parameters include:

\begin{itemize}
    \item RPC Endpoint: \texttt{https://rpc-amoy.polygon.technology/}
    \item Block Explorer: PolygonScan Amoy
    \item Average Block Time: $\sim$2 seconds
    \item Transaction Cost: $<$\$0.001 equivalent
\end{itemize}

\section{Implementation Details}

\subsection{Smart Contract Logic}

We implement four Solidity smart contracts, each serving distinct purposes within the credential ecosystem.

\subsubsection{SDCToken (ERC-20)}

The SDCToken contract implements a fungible reward token following the ERC-20 standard with extensions:

\begin{lstlisting}[language=Solidity,caption={SDCToken Core Structure}]
contract SDCToken is ERC20,
    ERC20Burnable, ERC20Pausable,
    AccessControl, ERC20Permit {

    bytes32 public constant MINTER_ROLE =
        keccak256("MINTER_ROLE");
    bytes32 public constant PAUSER_ROLE =
        keccak256("PAUSER_ROLE");

    uint256 public constant RSVP_REWARD = 10;
    uint256 public constant
        ATTENDANCE_REWARD = 20;

    mapping(address => uint256)
        public totalRewardsEarned;
    mapping(address => uint256)
        public rewardCount;
}
\end{lstlisting}

Key features include:
\begin{itemize}
    \item Role-based minting with MINTER\_ROLE
    \item Pausable functionality for emergency stops
    \item ERC-20 Permit for gasless approvals
    \item Built-in reward distribution functions
    \item Supply tracking and optional maximum cap
\end{itemize}

\subsubsection{AchievementSBT (Soulbound NFT)}

The AchievementSBT contract implements non-transferable achievement tokens:

\begin{lstlisting}[language=Solidity,caption={AchievementSBT Transfer Restriction}]
function _update(
    address to,
    uint256 tokenId,
    address auth
) internal override returns (address) {
    address from = _ownerOf(tokenId);

    if (from != address(0) &&
        to != address(0)) {
        require(!transferLock[tokenId],
            "Token is soulbound");
    }

    return super._update(to, tokenId, auth);
}
\end{lstlisting}

The soulbound mechanism prevents transfers between non-zero addresses while allowing initial minting (from zero address) and burning (to zero address). Each token stores:

\begin{itemize}
    \item Event identifier (bytes32 hash)
    \item Badge type (attendance, winner, volunteer)
    \item Issuance timestamp
    \item Metadata URI (IPFS reference)
    \item Transfer lock status
    \item Revocation status and reason
\end{itemize}

\subsubsection{Collectible721 (Tradable NFT)}

For achievements that should be tradable (referral rewards, limited editions), we implement a standard ERC-721 with scarcity controls:

\begin{lstlisting}[language=Solidity,caption={Collectible721 Scarcity Control}]
mapping(bytes32 => uint256)
    public maxSupply;
mapping(bytes32 => uint256)
    public currentSupply;

function mint(...) external
    onlyRole(MINTER_ROLE) {
    require(
        maxSupply[collectibleType] == 0 ||
        currentSupply[collectibleType] <
            maxSupply[collectibleType],
        "Max supply reached"
    );
    currentSupply[collectibleType]++;
    // ... minting logic
}
\end{lstlisting}

\subsubsection{SunDevilBadge (Legacy Contract)}

A simplified badge contract maintaining backward compatibility:

\begin{lstlisting}[language=Solidity,caption={Badge Metadata Structure}]
struct Badge {
    uint256 eventId;
    string eventName;
    string eventDate;
    string achievementType;
    string metadataURI;
    uint256 issuedAt;
    address issuer;
}
\end{lstlisting}

\subsection{Stakeholder Interactions}

The system supports four primary stakeholder types:

\textbf{Students:} Discover events, RSVP (earning 10 SDC), attend events (earning badges), view credentials in personal gallery, convert SDC to MATIC, and share achievements with employers.

\textbf{Event Organizers:} Create events, manage RSVPs, track attendance, and request badge issuance for attendees.

\textbf{Administrators:} Mint badges directly, distribute SDC rewards, manage user roles, and monitor system health.

\textbf{Verifiers (Employers/Faculty):} Access public verification portal to validate student credentials without authentication.

\subsection{Transactions and Events}

Smart contracts emit events for all significant state changes, enabling off-chain indexing and real-time updates:

\begin{lstlisting}[language=Solidity,caption={Key Smart Contract Events}]
// SDCToken Events
event RewardDistributed(
    address indexed recipient,
    uint256 amount,
    string rewardType,
    bytes32 indexed referenceId,
    uint256 timestamp
);

// AchievementSBT Events
event AchievementMinted(
    uint256 indexed tokenId,
    address indexed recipient,
    bytes32 indexed eventId,
    bytes32 badgeType,
    string metadataURI,
    uint256 timestamp
);

event AchievementRevoked(
    uint256 indexed tokenId,
    string reason,
    address revokedBy,
    uint256 timestamp
);
\end{lstlisting}

\subsection{Access Control}

We implement OpenZeppelin's AccessControl for granular permission management:

\begin{table}[htbp]
\caption{Role-Based Access Control Matrix}
\begin{center}
\begin{tabular}{|l|l|l|}
\hline
\textbf{Role} & \textbf{Permissions} & \textbf{Holder} \\
\hline
DEFAULT\_ADMIN & Grant/revoke roles & Deployer \\
\hline
MINTER\_ROLE & Mint tokens/NFTs & Backend \\
\hline
PAUSER\_ROLE & Pause contracts & Admin \\
\hline
REVOKER\_ROLE & Revoke badges & Admin \\
\hline
\end{tabular}
\label{tab:rbac}
\end{center}
\end{table}

Backend minting uses EIP-712 signatures for permit-based authorization, allowing the backend to authorize mints without holding tokens:

\begin{lstlisting}[language=Solidity,caption={EIP-712 Permit Minting}]
function mintWithPermit(
    address to,
    bytes32 eventId,
    bytes32 badgeType,
    string calldata metadataURI,
    uint256 deadline,
    bytes calldata signature
) external nonReentrant whenNotPaused {
    require(block.timestamp <= deadline,
        "Permit expired");

    bytes32 structHash = keccak256(
        abi.encode(
            MINT_TYPEHASH, to, eventId,
            badgeType, keccak256(
                bytes(metadataURI)),
            _useNonce(to), deadline
        )
    );

    // Verify signature and mint
}
\end{lstlisting}

\subsection{Immutability}

Blockchain immutability ensures that once a credential is issued:

\begin{enumerate}
    \item The issuance timestamp is permanently recorded
    \item The recipient address cannot be altered
    \item The event association is cryptographically linked
    \item Historical ownership is preserved in transaction logs
\end{enumerate}

While badges can be ``revoked,'' the original issuance record remains on-chain. Revocation adds a flag and reason without deleting the original data:

\begin{lstlisting}[language=Solidity,caption={Revocation Preserves History}]
function revoke(uint256 tokenId,
    string calldata reason) external
    onlyRole(REVOKER_ROLE) {
    require(_ownerOf(tokenId) != address(0),
        "Token does not exist");
    require(!isRevoked[tokenId],
        "Already revoked");

    isRevoked[tokenId] = true;
    revocationReason[tokenId] = reason;

    emit AchievementRevoked(tokenId, reason,
        msg.sender, block.timestamp);
}
\end{lstlisting}

\subsection{Storage Architecture}

We implement a hybrid storage model balancing privacy with transparency:

\textbf{On-Chain Storage:}
\begin{itemize}
    \item Token ownership mappings
    \item Event/badge type identifiers (hashed)
    \item Issuance timestamps
    \item Metadata URI references
    \item Role assignments
    \item Supply statistics
\end{itemize}

\textbf{Off-Chain Storage (SQLite):}
\begin{itemize}
    \item User credentials and authentication
    \item Event details (title, description, location)
    \item RSVP records
    \item Pre-claim SDC balances
    \item Transaction history references
\end{itemize}

The database schema includes ten tables with proper foreign key relationships:

\begin{lstlisting}[language=SQL,caption={Core Database Tables}]
CREATE TABLE users (
    id INTEGER PRIMARY KEY,
    username TEXT UNIQUE,
    password TEXT, -- bcrypt hashed
    email TEXT UNIQUE,
    role TEXT DEFAULT 'student',
    sdc_tokens INTEGER DEFAULT 100,
    wallet_address TEXT
);

CREATE TABLE minted_badges (
    token_id INTEGER PRIMARY KEY,
    student_wallet TEXT,
    event_id INTEGER,
    tx_hash TEXT,
    network TEXT,
    created_at DATETIME
);
\end{lstlisting}

\subsection{User Interface}

The frontend provides role-appropriate interfaces:

\textbf{Student Dashboard:} Displays SDC balance, earned badges, RSVP'd events, and conversion history. Integrates MetaMask for wallet connection.

\textbf{Event Portal:} Searchable event listings with filtering by category, campus, date, and availability. One-click RSVP with automatic SDC reward.

\textbf{Badge Gallery:} Visual display of earned credentials with metadata, verification links, and sharing options.

\textbf{Admin Panel:} Badge minting interface, reward distribution, user management, and system monitoring.

\textbf{Verification Portal:} Public interface for employers/faculty to verify credentials by wallet address or token ID.

\section{Results}

\subsection{Deployment Statistics}

We successfully deployed the smart contract suite to Polygon Amoy testnet:

\begin{table}[htbp]
\caption{Contract Deployment Results}
\begin{center}
\begin{tabular}{|l|c|c|}
\hline
\textbf{Contract} & \textbf{Gas Used} & \textbf{Size (KB)} \\
\hline
SDCToken & 2,847,321 & 12.4 \\
\hline
AchievementSBT & 3,156,892 & 14.7 \\
\hline
Collectible721 & 2,934,567 & 13.2 \\
\hline
SunDevilBadge & 1,856,234 & 8.9 \\
\hline
\end{tabular}
\label{tab:deployment}
\end{center}
\end{table}

\subsection{Transaction Performance}

Testing on Polygon Amoy demonstrated consistent performance:

\begin{itemize}
    \item Average transaction confirmation: 2.1 seconds
    \item Badge minting gas cost: $\sim$150,000 gas
    \item SDC distribution gas cost: $\sim$65,000 gas
    \item Batch minting (10 badges): $\sim$450,000 gas
\end{itemize}

\subsection{Functional Verification}

All smart contract functions were verified through comprehensive test suites:

\begin{lstlisting}[caption={Test Suite Results}]
SDCToken Tests
  Deployment (4 tests) - PASSED
  Minting (6 tests) - PASSED
  Reward Distribution (5 tests) - PASSED
  Access Control (4 tests) - PASSED

Total: 19 tests, 0 failures
\end{lstlisting}

\section{Analysis}

\subsection{Scalability}

Polygon's Layer 2 architecture provides significant scalability advantages:

\begin{itemize}
    \item \textbf{Throughput:} Up to 65,000 transactions per second
    \item \textbf{Block Time:} $\sim$2 seconds vs Ethereum's $\sim$12 seconds
    \item \textbf{Finality:} Near-instant for user experience, with periodic Ethereum checkpointing for security
\end{itemize}

For ASU's scale ($\sim$70,000 students), even with 100\% participation in daily events, the system can handle:
\[
\frac{70,000 \text{ badges/day}}{86,400 \text{ seconds/day}} \approx 0.81 \text{ TPS}
\]
This is well within Polygon's capacity, leaving significant headroom for growth.

\subsection{Gas Cost Analysis}

At current Polygon gas prices ($\sim$30 gwei), operational costs are minimal:

\begin{table}[htbp]
\caption{Operation Cost Analysis (Polygon)}
\begin{center}
\begin{tabular}{|l|c|c|}
\hline
\textbf{Operation} & \textbf{Gas} & \textbf{Cost (USD)} \\
\hline
Mint Badge & 150,000 & \$0.0045 \\
\hline
Distribute SDC & 65,000 & \$0.002 \\
\hline
Batch Mint (10) & 450,000 & \$0.014 \\
\hline
Transfer SDC & 45,000 & \$0.0014 \\
\hline
\end{tabular}
\label{tab:costs}
\end{center}
\end{table}

Annual operational costs for 10,000 badge issuances would be approximately \$45---a negligible expense compared to traditional paper certificate costs.

\subsection{Data Management}

The hybrid architecture optimizes for both efficiency and compliance:

\textbf{Advantages:}
\begin{itemize}
    \item Sensitive PII remains off-chain
    \item GDPR ``right to erasure'' applicable to database records
    \item On-chain records contain only hashed identifiers
    \item Metadata stored on IPFS for content availability
\end{itemize}

\textbf{Tradeoffs:}
\begin{itemize}
    \item Requires maintaining database availability
    \item Token metadata URIs must remain accessible
    \item Event details require off-chain lookup
\end{itemize}

\subsection{Privacy and Regulatory Implications}

Our design addresses key regulatory requirements:

\textbf{FERPA Compliance:} Educational records remain in the university database. On-chain data contains only cryptographic hashes and timestamps, which do not constitute protected educational records.

\textbf{GDPR Considerations:} User wallet addresses may be considered personal data. Our design allows users to generate new wallets, providing pseudonymity. The immutable nature of blockchain conflicts with ``right to erasure,'' addressed by storing only non-identifying hashes on-chain.

\textbf{Data Minimization:} On-chain storage is limited to essential verification data. Detailed event information and user profiles remain off-chain.

\subsection{Comparison with Traditional Implementation}

\begin{table}[htbp]
\caption{Blockchain vs Traditional Credential Systems}
\begin{center}
\begin{tabular}{|p{2cm}|p{2.5cm}|p{2.5cm}|}
\hline
\textbf{Aspect} & \textbf{Traditional} & \textbf{Blockchain} \\
\hline
Verification & Contact issuer & Self-service, instant \\
\hline
Fraud Risk & Forgery possible & Cryptographically secure \\
\hline
Availability & Issuer-dependent & 24/7, decentralized \\
\hline
Portability & Physical copies & Digital, global access \\
\hline
Cost & \$5-20/certificate & $<$\$0.01/badge \\
\hline
Revocation & Manual process & Smart contract call \\
\hline
Audit Trail & Limited & Complete, immutable \\
\hline
\end{tabular}
\label{tab:comparison}
\end{center}
\end{table}

\section{Future Work}

Several enhancements are planned for future development:

\subsection{Technical Improvements}
\begin{itemize}
    \item Migration to Polygon mainnet for production deployment
    \item Integration with IPFS/Arweave for decentralized metadata storage
    \item Implementation of zero-knowledge proofs for selective disclosure
    \item Mobile application development (React Native)
    \item Integration with university SSO (Shibboleth/SAML)
\end{itemize}

\subsection{Feature Enhancements}
\begin{itemize}
    \item Cross-institution credential recognition
    \item Employer verification API
    \item Badge stacking for compound credentials
    \item Integration with LinkedIn and professional networks
    \item Governance token for community decisions
\end{itemize}

\subsection{Research Directions}
\begin{itemize}
    \item Analysis of gamification impact on participation rates
    \item Study of employer adoption of blockchain credentials
    \item Development of credential interoperability standards
\end{itemize}

\section{Individual Contributions}

\begin{table}[htbp]
\caption{Team Member Contributions}
\begin{center}
\begin{tabular}{|p{2cm}|p{5.5cm}|}
\hline
\textbf{Member} & \textbf{Contributions} \\
\hline
Member 1 & Smart contract development (SDCToken, AchievementSBT), deployment scripts, testing framework \\
\hline
Member 2 & Backend development (Node.js/Express), database design, API implementation \\
\hline
Member 3 & Frontend development (HTML/CSS/JS), user interface design, MetaMask integration \\
\hline
Member 4 & System integration, wallet services, documentation, project management \\
\hline
\end{tabular}
\label{tab:contributions}
\end{center}
\end{table}

\section{Conclusion}

SunDevilSync 2.0 demonstrates the practical application of blockchain technology for educational credential verification. By combining soulbound NFTs for non-transferable achievements with a gamified token reward system, we create an engaging platform that incentivizes student participation while producing verifiable, tamper-proof credentials.

The hybrid architecture balances the transparency benefits of blockchain with privacy requirements, storing sensitive data off-chain while maintaining immutable credential records on-chain. Deployment on Polygon provides cost-effective, high-throughput transaction processing suitable for educational institution scale.

Our implementation provides a foundation for broader adoption of blockchain-based credentials in higher education, with potential for cross-institution recognition and employer integration. The open-source nature of the project enables adaptation for other universities and credential use cases.

\section*{Acknowledgments}

We thank Arizona State University for supporting this project and the Polygon team for providing testnet resources. We also acknowledge the OpenZeppelin team for their audited smart contract libraries.

\begin{thebibliography}{00}
\bibitem{acfe2022} Association of Certified Fraud Examiners, ``Report to the Nations: 2022 Global Study on Occupational Fraud and Abuse,'' 2022.

\bibitem{mit2019} MIT Digital Credentials Consortium, ``Building the Digital Credential Infrastructure for the Future,'' MIT Media Lab, 2019.

\bibitem{ebsi2021} European Commission, ``European Blockchain Services Infrastructure: Diploma Use Case,'' EBSI Documentation, 2021.

\bibitem{openbadges} Mozilla Foundation, ``Open Badges Specification,'' IMS Global Learning Consortium, 2018.

\bibitem{buterin2022} E. G. Weyl, P. Ohlhaver, and V. Buterin, ``Decentralized Society: Finding Web3's Soul,'' SSRN, 2022.

\bibitem{polygon2021} Polygon Technology, ``Polygon Whitepaper: A Scalable Multi-Chain System,'' 2021.

\bibitem{hamari2014} J. Hamari, J. Koivisto, and H. Sarsa, ``Does Gamification Work? A Literature Review of Empirical Studies on Gamification,'' in Proc. 47th Hawaii Int. Conf. System Sciences, 2014.

\bibitem{openzeppelin} OpenZeppelin, ``OpenZeppelin Contracts: Secure Smart Contract Library,'' \url{https://openzeppelin.com/contracts/}, 2024.

\bibitem{eip721} W. Entriken, D. Shirley, J. Evans, and N. Sachs, ``EIP-721: Non-Fungible Token Standard,'' Ethereum Improvement Proposals, 2018.

\bibitem{eip20} F. Vogelsteller and V. Buterin, ``EIP-20: Token Standard,'' Ethereum Improvement Proposals, 2015.

\bibitem{eip712} R. Floersch and D. Finlay, ``EIP-712: Typed Structured Data Hashing and Signing,'' Ethereum Improvement Proposals, 2017.

\bibitem{solidity} Ethereum Foundation, ``Solidity Documentation,'' \url{https://docs.soliditylang.org/}, 2024.

\end{thebibliography}

\end{document}
